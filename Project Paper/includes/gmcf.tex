\begin{figure}
    \includegraphics[width=0.5\textwidth]{graphs/gmcfArchitecture.png}
    \caption{GMCF Architecture}
    \label{fig:gmcfArchitecture}
\end{figure}

GMCF is a relatively new framework with development beginning in 2014. GMCF's
primary function is for model coupling. One of long term goals of GMCF is to
automate as much of the model coupling process as possible. Current frameworks
currently require hand modification of existing model code and the writing of
additional code to describe and control the communication between models. This
can present a level of difficulty that a research team may not be willing to
accept so a framework that can automate a significant amount of the model
coupling work would be welcomed.

GMCF also aims to change the way model coupling is organised at runtime. The
current model coupling frameworks are focussed on clusters with single-core
machines and do not take into account the uptake in multicore and heterogeneous
nodes in modern clusters. The focus on single core nodes leads to a
communication that is necessarily distributed using MPI or an equivalent
framework and, as such, load balancing between models is organised in such a way
that each model gets a fixed number of nodes, proportional to its runtime
compared with the runtimes of the other models. With the advent of multicore
clusters, a new load balancing technique can be implemented, which is what GMCF
does. GMCF aims to limit the communication of models to processes within a
single node, with limited distributed memory communication. The idea behind this
is that communication in the cluster is localised to within each shared memory
node with only occasional communication between nodes. This also means that each
node can balance its load more evenly for improved performance since each node
now executes part of each individual model.

To date, GMCF has been used to couple the LES with WRF, acting as a proof of
concept to the new ideas behind the framework.

GMCF works by having a concept of packets which are transferred between the
source and destination threads, arriving at the destination thread's FIFO queue.
Each thread has a main FIFO queue for receiving packets and packets are
demultiplexed from this queue into another queue dependent on the packet's type
and source thread ID. Figure~\ref{gmcfArchitecture} shows this architecture from
the perspective of a single GMCF Tile, which is owned by a single thread.

GMCF has many packet types, including a packet type to request data, a packet
type to respond with a pointer to the requested data, and an acknowledgement
packet type to let the source thread know that the data has been copied by the
destination so the array being sent is safe to use.

Climate scientists create independent simulations that each model a different
part of the planet's ecosystem. To model more complex aspects of the ecosystem,
and for more accurate results, multiple simulations are required to interact and
share data.

There are three ways of having models work together \cite{Thevenin}. The first
is merging the model code bases together. This generates a single executable
with the biggest benefit being that all models have efficient data exchanges
since all data is contained within the one memory space. However, merging code
bases is not an ideal solution because the merge itself is difficult to complete
and further development of the individual models is now more difficult. The
problems associated with merging code bases grow exponentially as the number of
models in the coupling grows.

The second is to use a communication library directly, such as MPI. The main
benefit of this over merging code bases is that each model can keep its own
independent code base with only explicit communication code being added to each
model. This solution is still not ideal since the communication code is not
generic. Every model coupling configuration will require its own communication
code and coupling more than two models together results in very complicated
exchanges needing to be implemented.

The third solution is to use a model coupling library to abstract the low level
communication complexities and act as a ``middle man'' between each model.
Techniques to facilitate this vary however they all generally require some code
changes for framework initialisation and to register arrays as being part of the
inter-model communication. A plain text file then handles the model coupling
specifics. This method of model coupling is the most popular with, for example,
the LES being coupled to the Weather Research and Forecasting Model (WRF) using
a number of model coupling libraries. This has led to more accurate wind
velocities being predicted compared with the predictions from WRF alone
\cite{Kinbara2010,Nakayama1998}, showing the benefit of having multiple models
interact with one another. There are many model coupling frameworks available,
including: the Model Coupling Toolkit (MCT) \cite{Jacob2005,Larson2005}; OASIS
\cite{Valcke2013,Valcke}; the Earth System Modeling Framework (ESMF)
\cite{Ramework2004}; and OpenPALM \cite{Piacentini2011}.

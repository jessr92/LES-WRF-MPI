Individuals models for ocean, land, atmosphere, etc are useful in ther own right
however climate scientists are now wanting to model more complex scenarios that
require collaboration from multiple models. This collaboration can also lead to
more accurate results since the additional effects from other models can
influence the results of the other models in the overall system. For example,
coupling WRF and LES led to greater wind velocity being predicted compared with
running WRF on its own \cite{Kinbara2010,Nakayama1998}.

Currently, WRF and LES are independent models and LES has two variants: the
original single threaded code and the OpenCL-accelerated variant
\cite{Vanderbauwhede2014}. Only WRF is currently able to make use of a
distributed memory system. Vanderbauwhede and Takemi \cite{Vanderbauwhede2013}
have investigated the benefits of GPU accelerating WRF and found this to be
``feasible and worthwhile'' however this is out of scope for this proposal.

The problem to be addressed is two-fold: creating a MPI variant of LES to allow
LES to run on a distributed memory system and then coupling this variant of LES
with WRF. The MPI variant of the LES will increase the performance of the
simulator since it will allow it to make use of the resources on a multi-node
distributed memory system such as a Beowulf cluster. The coupling of MPI LES
with WRF will create a system that benefits from scalable performance and high
accuracy results.

The traditional method of coupling faces the \textit{M by N problem} (discussed
later) where a each node executes one model only and thus all data exchanges are
between nodes. This is stressful on the network bandwidth so GMCF takes a
different approach. GMCF has each node in a cluster run each model's executable
with the intention that most data exchanges are within a single node which will
be higher performing than exchanges over a physical network cable. The
comparison of GMCF to traditional model couplers will shed light on the benefits
and drawbacks of this approach.

The ecosystem of our planet is highly complex. Climate scientists that are
interested in modelling this ecosystem have created a number of independent
systems which each model one aspect, for instance a Large Eddy Simulator to
study turbulent air flows \cite{Nakayama2011,Nakayama2012} and the Weather
Research and Forecasting Model for mesoscale weather prediction
\cite{Michalakes2000}. For the last decade or so, interest has evolved from
individual models to combining models \cite{Michalakes2010} to obtain more
accurate results and to model more complex aspects of our planet's ecosystem.
With this trend, a number of model coupling frameworks have been developed to
support the desire for co-simulation. These model couplers allow a single
application to make use of multiple models without significant changes to the
code bases of the individual models. Models communicate with each other via data
exchanges at each timestep which are handled by the coupling framework. The
coupling frameworks typically use MPI for communication since MPI is the
\textit{de facto} standard for parallel communication.

The proposed work for the project involves three distinct stages. The first
requires the modification of the existing LES code to create a distributed
version that makes use of MPI to allow performance to scale on large computer
clusters. An additional aim for this task is to facilitate the use of the
OpenCL-accelerated LES \cite{Vanderbauwhede2014} in the distributed system to
improve the model's throughput further on heterogeneous systems. The second task
of the first stage involves coupling the distributed LES with WRF using a model
coupling framework. The second stage involves a performance evaluation of both
the distributed LES and the LES-WRF coupling. The third stage involves comparing
the coupling from the first stage to a coupling using the Glasgow Model Coupling
Framework (GMCF) \cite{Vanderbauwhede2014}.

A number of modeling frameworks have been looked at: the Model Coupling Toolkit
(MCT) \cite{Larson2005,Jacob2005}; OASIS (specifically OASIS3-MCT)
\cite{Valcke,Valcke2013}; the Earth System Modeling Framework (ESMF)
\cite{Ramework2004}; and OpenPALM \cite{Piacentini2011}. After comparing these
model coupling frameworks, OASIS3-MCT was chosen as the preferred framework
given its continued development and simple coupling mechanism.

The rest of the document is structured as follows:
Section~\ref{sec:statementOfProblem} discusses the research problem;
Section~\ref{sec:literatureReview} is an in-depth literature review for the
state of the art in distributed and coupled applications;
Section~\ref{sec:proposedApproach} contains the proposed approach for working on
the research problem and Section~\ref{sec:projectPlan} describes the work plan
for the approach including estimated deliverable dates.

Our planet's ecosystem is highly complex. Geoscientists that are interested in
modelling this ecosystem have created numerous systems which each model one
aspect, for instance a Large Eddy Simulator (LES) to study turbulent air flows
\cite{Nakayama2011,Nakayama2012} and the Weather Research and Forecasting Model
(WRF) for mesoscale weather prediction \cite{Michalakes2000}. For the last
decade or so, interest has evolved from individual models to combining models
\cite{Michalakes2010} to obtain more accurate results and to model more complex
aspects of our planet's ecosystem. With this trend, a number of model coupling
frameworks have been created to support the desire for co-simulation. These
model couplers allow a single application to make use of multiple models without
significant changes to the individual models' code bases. Models communicate
with each other via data exchanges at each timestep which are handled by the
coupling framework. The coupling frameworks typically use the Message Passing
Interface (MPI) for communication since MPI is the \textit{de facto} standard
for parallel communication.

The proposed work for the project involves three distinct stages. The first
requires the modification of the existing LES code to create a distributed
version that makes use of MPI to allow performance to scale on large computer
clusters. An additional aim of the first stage is to facilitate the use of the
OpenCL-accelerated LES \cite{Vanderbauwhede2014} in the distributed system to
improve the model's throughput further on heterogeneous systems. The second
stage involves coupling the distributed LES with WRF using a model coupling
framework. The third stage involves comparing the coupling from the second stage
to a coupling using the Glasgow Model Coupling Framework (GMCF)
\cite{Vanderbauwhede2014}.

A number of modeling frameworks have been looked at, including: the Model
Coupling Toolkit (MCT) \cite{Larson2005,Jacob2005}; OASIS (specifically
OASIS3-MCT) \cite{Valcke,Valcke2013}; the Earth System Modeling Framework (ESMF)
\cite{Ramework2004}; and OpenPALM \cite{Piacentini2011}. After comparing these
model coupling frameworks, OASIS3-MCT was chosen as the preferred framework
given its continued development and simple coupling mechanism.

The remainder of this document discusses the research problem in detail before
going into an in-depth literature review for the state of the art in distributed
and coupled applications. The proposed approach and work plan for the project is
then given to conclude the proposal.

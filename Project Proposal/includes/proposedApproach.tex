As stated before, there are a number of distinct stages to the project and each
stage requires a different approach.

The first stage involves creating a distributed version of the LES. This will
use MPI since it is the \textit{de facto} standard for cross-process
communication and OASIS3-MCT uses MPI1 capabilities for the coupling itself.
Even though OASIS3-MCT itself is limited to MPI1 capabilities, there should be
no issue using MPI2, or even MPI3 capabilities for the model-local parallelism
since MPI is mostly backward compatible. The dummy OASIS3-MCT coupling created
as part of the feasibility study was compiled with MPICH 3.1.2, an MPI3 capable
library, without issue. It is thought that the overhead caused by the MPI data
exchanges should not be too significant and should scale for tens of nodes. It
may be the case that boundary exchanges will have to be limited to once per
timestep, rather than the four exchanges per timestep currently in place for the
open and periodic boundaries. This could reduce the accuracy of the results
calculated however any differences should not be significant compared to the
single threaded case.

The first stage's second task involves coupling LES with WRF. For each timestep
of WRF, LES will run about one hundred times before data is exchanged since
WRF's calculations take significantly longer than LES's. This will need to be
expressed in the coupling mechanism to avoid unnecessary data transfers and
minimal requirement for I/O operations. There are also potential problems with
modifying WRF for the OASIS coupling since WRF is a highly complex piece of
software and thus making changes, especially with a lack of experience with the
code base, could be challenging.

The second stage is the evaluation of performance which will have two parts:
evaluating the MPI LES and the coupled system performance as node count
increases, perhaps with different MPI implementations (say MPICH and OpenMPI).
MPI libraries aren't strictly interchangeable however with the availability of
the wrapper compiler \textit{mpifort} mixing MPI implementations may be
straightforward.

The third stage involves comparing the differences in performance of the
OASIS3-MCT coupling to the GMCF coupling. GMCF is in a proof-of-concept stage
however has already been used to couple OpenMP WRF with the OpenCL accelerated
LES variant successfully.

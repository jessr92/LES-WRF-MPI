\documentclass[a4paper,twocolumn,10pt]{article}

\usepackage[cm]{fullpage}

\usepackage{titlesec}
\titlespacing{\section}{0pt}{0pt}{0pt}
\titlespacing{\subsection}{0pt}{0pt}{0pt}
\titlespacing{\subsubsection}{0pt}{0pt}{0pt}

\title{Coupling the distributed Large Eddy Simulation (LES) and Weather
Research and Forecasting model (WRF) using OASIS3-MCT and MPI}

\author{\large\textsc{Gordon Reid - 1002536r}\\University of Glasgow}

\date{\today}

\begin{document}
\setlength{\parskip}{0.3cm}

\maketitle

\begin{abstract}

\noindent Scientists studying the geosciences have created software which models
different aspects of our planet's ecosystem, for example ocean, wind, land and
atmosphere. On their own, each model is useful however for higher accuracy, and
to model more complex aspects of our climate, these models need to be coupled
together. The project proposed involves creating a distributed version of the
Large Eddy Simulator (LES) then coupling the LES with the Weather Research and
Forecasting model (WRF). The coupled system should scale to large computing
clusters. Ideally the coupling would also be generic such that the
OpenCL-accelerated LES could be used to allow greater performance on
heterogeneous systems.

\end{abstract}

\section*{Introduction}

Our planet's ecosystem is highly complex. Geoscientists that are interested in
modelling this ecosystem have created numerous systems which each model one
aspect, for instance a Large Eddy Simulator (LES) to study turbulent air flows
\cite{Nakayama2011,Nakayama2012} and the Weather Research and Forecasting Model
(WRF) for mesoscale weather prediction. For the last decade or so, interest has
evolved from individual models to combining models \cite{Michalakes2010}. With
this trend, a number of model coupling frameworks have been created to support
the desire for co-simulation. These allow a single application to make use of
multiple models without significant changes to the individual models code bases.
Models communicate with each other via data exchanges at each timestep.

The proposed work for the project involves two distinct stages. The first
requires the modification of existing LES code to create a distributed version
using the Message Passing Interface (MPI) for use on computer clusters. An
additional aim is to allow use of the OpenCL-accelerated LES
\cite{Vanderbauwhede2014} in the distributed system to improve the model's
throughput further. The second stage involves the coupling of the distributed
LES with the WRF using a model coupling framework.

A number of modeling frameworks have been looked at, including: The Model
Coupling Toolkit (MCT) \cite{Larson2005}; OASIS3-MCT \cite{Valcke,Valcke2013};
the Earth System Modeling Framework (ESMF) \cite{Ramework2004}; and OpenPALM
\cite{Piacentini2011}. After comparing these modeling frameworks, OASIS3-MCT was
chosen given its continued development and simple coupling mechanism.

The remainder of the document discusses the research problem in detail before
going onto an in-depth literature review for the state of the art in distributed
and coupled applications. The proposed approach and work plan for the project is
then given with a summary afterwords to conclude the proposal.

\section*{Statement of Problem}

\section*{Literature Review}

\section*{Proposed Approach}

\section*{Work Plan}

\section*{Summary}

\bibliographystyle{plain}
\bibliography{1002536r}

\end{document}
